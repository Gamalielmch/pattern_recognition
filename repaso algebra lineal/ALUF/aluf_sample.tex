\documentclass{beamer}
\usetheme{ALUF}

\usepackage[utf8]{inputenc}
% \usepackage{palatino}
% \usepackage[T1]{fontenc}
\usepackage{lmodern}
\usepackage[expert]{mathdesign}
\usepackage[protrusion=true,expansion=true,tracking=true,kerning=true]{microtype}



\title{Gibbs Sampling for the Un-initiated}
\subtitle{As if this needs a subtitle}
\author{Prof. Dr. FirstName LastName}
\date{\today}
\institute{\url{email@some-cool-place.ext}\\\url{http://www.cool-url.com}}

\begin{document}

\begin{frame}[plain,t]
\titlepage
\end{frame}

\begin{frame}% [plain,t]
	\frametitle{Outline}
\tableofcontents
\end{frame}

%=============================================================================================

\section{Introduction}
\begin{frame}
\frametitle{Some awesome frame title but not too long}
\framesubtitle{That is what the subtitle is for}
\begin{itemize}
\item First thing
	\begin{itemize}
	\item small point
	\item fine print
	\end{itemize}
\item Second thing
	\begin{enumerate}
	\item point 1
	\end{enumerate}
\item Third thing
	\begin{description}
	\item[Research] the scientific pursuit for knowledge
	\end{description}
\end{itemize}
\end{frame}
%
%\subsection{Awesome subsection}
%\begin{frame}
%	\frametitle{Another Frame Title}
%
%	Here comes some math!
%
%	\begin{equation}
%
%		\begin{bmatrix}
%	        \Phi_t \\
%	        \Phi_{t+1} \\
%	        \vdots \\
%	        \Phi_{t+H}
%	    \end{bmatrix}
%	    ~=~
%	    \begin{bmatrix}
%	        \phi_t^1, \ldots, \phi_t^d \\
%	        \phi_{t+1}^1, \ldots, \phi_{t+1}^d \\
%	        \vdots \\
%	        \phi_{t+H}^1, \ldots, \phi_{t+H}^d
%	    \end{bmatrix}
%
%		\label{eq:random}
%	\end{equation}
%
%\end{frame}

\subsection{Some nice subsection}
\begin{frame}
\frametitle{Blocks}
\begin{definition}[Greetings]
Hello World
\end{definition}

\begin{theorem}[Fermat's Last Theorem]
$a^n + b^n = c^n, n \leq 2$
\end{theorem}

\begin{alertblock}{Uh-oh.}
By the pricking of my thumbs.
\end{alertblock}

\begin{exampleblock}{Uh-oh.}
Something evil this way comes.
\end{exampleblock}

\end{frame}


\section{Another Section} % (fold)
\label{sec:another_section}

\begin{frame}
	\frametitle{Notation}
	\begin{definition}[Random Variable]
		Consider $\Omega, F, \mu$, with $\Omega$ being the set of events, $F$ the $\sigma$-algebra on $\Omega$ and some arbitrary measure $\mu$. Further consider an observation space $\Omega', F', \mu'$... A random variable is a deterministic function that 'transports/maps' events from $\Omega$ to $\Omega'$ and effectively induces a new measure $\mu'$. When $\mu'(\Omega') = 1$, it is a probability measure.

	\end{definition}
\end{frame}


% section another_section (end)


%\ThankYouFrame

\end{document}
